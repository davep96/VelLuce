\documentclass[a4paper,11pt]{article}

\usepackage[T1]{fontenc}

\usepackage[utf8]{inputenc}

\usepackage[italian]{babel}

\usepackage{graphicx}

\usepackage{indentfirst}

\usepackage{amsmath,amssymb}

\usepackage{enumitem} 

\newcommand{\virgolette}[1]{``#1''}

\usepackage[margin=1in]{geometry} %Smaller margins

\usepackage{lmodern} %Vector PDF

\usepackage{siunitx}

\usepackage{xcolor}

\usepackage{colortbl}

\usepackage{multirow}

\usepackage{rotating}

\usepackage{booktabs}

\usepackage{longtable}

\usepackage{graphicx}
\graphicspath{ {../../Immagini/} }

\usepackage{wrapfig}

\usepackage{siunitx} % Per unit� di misura in generale e la corretta rappresentazione dei numeri.

\usepackage{gensymb} % Per il simbolo di gradi

\usepackage{float}

\begin{document}
\section{Strumentazione}

Per estrarre sperimentalmente i termini necessari a calcolare la velocità della luce sono stati utilizzati i seguenti strumenti:

\begin{description}[align=left]
	
	\item[Laser] Una sorgente luminosa monocromatica di lunghezza d'onda $632.8\, \si{\nano\meter}$.
	\item[Lenti] Due lenti $L_1$ $L_2$ con focali rispettivamente di $48\,\si{\milli\meter}$ e $252\,\si{\milli\meter}$.
	\item[Squadra Forata] Una lastra bianca con un piccolo foro che consente il passaggio del fascio luminoso.
	\item[Specchi] Tre specchi di cui due piani e uno concavo con raggio di curvatura analogo alla distanza che percorre la luce quando viene compiuto l'esperimento.
	\item[Doppia Lamina Polaroid] Una doppia schermatura per diminuire l'intensità della luce durante la calibrazione dell'apparato sperimentale.
	\item[Supporto] Un supporto magnetico su cui appoggiare gli strumenti in maniera tale che questi restino stabili nella posizione in cui sono messi. In questo senso questo supporto presentava una scala graduata con sensibilità del millimetro.
	\item[Specchio Rotante] Uno specchio rotante collegato tramite cinghia a un motore che avvia la rotazione in senso orario o antiorario.
	\item[Motore] Un sistema di avviamento della cinghia collegata allo specchio rotante. Lo strumento pu\'o essere utilizzato sia per generare una rotazione sia in senso orario che in senso antiorario, ed è dotato di un display dove leggere il numero di Hz a cui il sistema è fatto ruotare, cos\'i come di un pulsante che fa ruotare il sistema alla frequenza massima di \SI{1500}{\hertz} circa.
	\item[Microscopio] Un microscopio attaccato a un nonio della sensibilità di $\SI{10}{\micro\meter}$.
	\item[Splitter] Una lastra di vetro semiriflettente che devia parte del fascio luminoso verso il microscopio.
	\item[Metro] Utilizzato per misura della distanza percorsa dalla luce durante una esecuzione dell'esperimento.
\end{description}
	

\section{Procedura Sperimentale}

La procedura sperimentale per effettuare delle misure di $c$ pu\'o essere suddivisa in due parti principali: la calibrazione dell'apparato sperimentale e le procedure di misura vere e proprie.

\subsection{Messa a Punto e Calibrazione}

La procedura di messa a punto dell'apparato sperimentale è stata eseguita secondo i passaggi che seguono:

\begin{description}
	\item[Misura delle Distanze] In primo luogo sono state misurate le distanze che separano gli specchi colpiti dal fascio luminoso. In questo modo noto il valore dell'indice di rifrazione dell'aria è possibile determinare il cammino ottico percorso dalla luce.
	\item[Verifica dell'Incidenza della Luce] Tramite la squadra forata si è verificato che la luce andasse a colpire lo specchio rotante.
	\item[Autocollimazione] Facendo ruotare lo specchio verificare che il fascio riflesso sia centrato con il foro di uscita del laser. Per questa operazione è stata rimossa la squadretta.
	\item[Messa a Punto delle Lenti] La lente $L_1$ è stata posizionata rispettivamente a $\SI{70}{\milli\meter}$ sul supporto servendosi della scala graduata e avendo cura di mantenere un orientamento corretto (a L rovesciata). La messa a punto viene completata orientando la lente $L_1$ in maniera tale che il fascio originato dal laser sia centrato sul foro della squadretta. Successivamente la lente $L_2$ è stata disposta a $378mm$ sulla scala. Queste posizioni sono necessarie per focalizzare il fascio in maniera corretta e ottenere contemporaneamente un \textit{waist} pi\'u piccolo possibile. La figura \ref{photo4} mostra la disposizione delle lenti e dello splitter sul supporto.
	\item[Messa a Punto dello Splitter] Posizionare il supporto contenente lo splitter (senza microscopio attaccato) alla distanza di $180mm$ sulla scala graduata. Inclinare la lente dello splitter in maniera tale che il fascio di luce riflesso dallo specchio rotante sia orientato verso la zona in cui andrà messo l'oculare. Filtrare la luce con le lamine polaroid e posizionare l'oculare avendo cura di effettuare una corretta messa a fuoco.
	\item[Messa a Punto degli Specchi] Orientare lo specchio rotante in maniera tale che la luce incidente venga riflessa contro il primo specchio piano. Tramite le viti micrometriche di questo specchio orientare il fascio contro il secondo specchio piano e di nuovo contro lo specchio concavo. Eseguire questa procedura anche per il fascio riflesso dallo specchio concavo, aggiustando opportunamente le viti micrometriche in maniera tale che i fasci di luce di \virgolette{andata} e \virgolette{ritorno} collimino con la maggior precisione possibile.
	\item[Regolazione del Waist] Spostando la lente $L_2$ lungo il supporto verificare che sull'oculare le dimensioni del \textit{waist} prodotto dal fascio lumisono sono il pi\'u piccolo possibile.
\end{description}

\begin{figure}[h]
		\centering
		\includegraphics[width = 0.98 \textwidth]{photo4}    
		\caption{Disposizione dello splitter e delle lenti sulla scala graduata.}\label{photo4}
	\end{figure}
\subsection{Procedure di Misura Sperimentale}
La presa dati consiste nella raccolta di una serie di differenze di distanze relative alla posizione del \textit{waist} sull oculare  -- opportunamente dotato di un mirino a X per centrare il \textit{waist} stesso, in quanto la posizione del punto rosso è dipendente dalla frequenza con cui lo specchio rotante viene fatto oscillare, come mostra l'equazione \ref{eqn:c}. La procedura sperimentale di misura consiste in una ripetizione dei seguenti passi:

\begin{itemize}
	\item Accendere il motore a una frequenza relativamente bassa ($50-100$ Hz) e centrare la posizione del \textit{waist} muovendo la vite micrometrica.
	\item Registrare il valore della posizione individuato dal nonio collegato alla vite micrometrica.
	\item Aumentare la frequenza di rotazione del motore (da $900-1000$ Hz) fino anche al massimo, e centrare la posizione del \textit{waist} muovendo la vite micrometrica.
	\item Di nuovo registrare il valore della posizione tramite il nonio.
\end{itemize}

Questa ripetizione va eseguita per entrambi i sensi in cui il motore è in grado di imprimere rotazione alla cinghia (\textit{clockwise} e \textit{counterclockwise}). Inoltre è possibile eseguire delle misure anche tra massimo \textit{clockwise} e massimo \textit{counterclockwise} avendo cura di fermare il motore senza passare immediatamente da un senso di rotazione all'altro. La ripetizione di misure indipendenti tra loro consentirà poi di eseguire analisi statistica sui campioni di dati estratti.

In questa sede va fatto notare un problema riscontrato durante le operazioni di misura: non è stato possibile individuare se il problema fosse relativo al motore, o al display che segnalava la frequenza di rotazione, ma quello che si osservava era la poca stabilità nel numero di Hz segnalati. Come conseguenza di questo fatto è risultato difficile eseguire 20-30 misure per ogni set di dati (specie considerando che la massima frequenza del motorino aveva un'autonomia di un minuto circa e spesso il display non si era ancora stabilizzato). Abbiamo fatto in modo di effettuare il maggior numero di misure possibili compatibilmente con i tempi richiesti per effettuare una misura (spesso oltre un minuto per attendere che il display fosse stabile).

\end{document}