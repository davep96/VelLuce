\documentclass[a4paper,11pt]{article}

\usepackage[T1]{fontenc}

\usepackage[utf8]{inputenc}

\usepackage[italian]{babel}

\usepackage{graphicx}

\usepackage{indentfirst}

\usepackage{amsmath,amssymb}

\usepackage{enumitem} 

\newcommand{\virgolette}[1]{``#1''}

\usepackage[margin=1in]{geometry} %Smaller margins

\usepackage{lmodern} %Vector PDF

\usepackage{siunitx}

\usepackage{xcolor}

\usepackage{colortbl}

\usepackage{multirow}

\usepackage{rotating}

\usepackage{booktabs}

\usepackage{longtable}

\usepackage{graphicx}
\graphicspath{ {../../Immagini/} }

\usepackage{wrapfig}

\usepackage{siunitx} % Per unit� di misura in generale e la corretta rappresentazione dei numeri.

\usepackage{gensymb} % Per il simbolo di gradi

\begin{document}
	
	\begin{equation}\label{eqn:D}
	D = D_1 + D_2 + D_3 \quad \sigma_D = \sqrt{\sigma_{D_1}^2 + \sigma_{D_2}^2 + \sigma_{D_3}^2}
	\end{equation}
	
	\begin{equation}\label{eqn:a}
	a = s_{spr} - s_{f_2} \quad \sigma_a = \sqrt{ \sigma_{s_{spr}}^2+\sigma_{s_{f_2}}^2}
	\end{equation}
	
	Da mettere da qualche parte!
	\begin{equation}\label{eqn:c}
	c = \dfrac{4 f_2 D^2 \left(\omega - \omega_0 \right)}{\left(D + a - f_2\right)\Delta\delta}
	\end{equation}
	Per una corretta visualizzazione dei numeri, con le unità di misura e gli ordini di grandezza usare il pacchetto siunitx con i comandi:\\
	
		\SI{5}{\centi\meter}
	\\
		\SI{633e-9}{\meter}
	\\
		\num{6.022e23}
	\\
	Inoltre usiamo la convenzione internazionale per cui i decimali si separano con un punto per piacere.
	

	
\end{document}
